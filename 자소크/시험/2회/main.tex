\documentclass{article}

\usepackage{kotex}
\usepackage{mdframed}
\usepackage{emoji}

\title{\textsc{2-Tiúspens dé Zasospikavaisausentrah}}
\author{제2회 자소크어 능력 평가}
\date{Zasospika d'Uagize-Lie, 2022. 8. 1}

\newmdenv[
  skipabove=\topsep,
  skipbelow=\topsep,
]{reminder}

\begin{document}

\maketitle

\paragraph{1.}
밑줄 친 부분과 발음이 같지 않은 것은?

\begin{reminder}
    Vlazif lo Navlazifàsiç Édiçoza\underline{que}
\end{reminder}

\begin{enumerate}
    \item mo\underline{que}
    \item ni\underline{qu}iks
    \item \underline{qu}ir
    \item si\underline{qu}er
    \item sio\underline{que}das
\end{enumerate}

\paragraph{2.}
빈칸에 들어갈 글자를 모두 조합하여 만들 수 있는 낱말의 의미는?

\begin{center}
    \begin{tabular}{|c|c|c|c|c|}
        \hline
        \huge \emoji{cook} & \huge \emoji{fist} & \huge \emoji{arrow-right} & \huge \emoji{billed-cap} & \huge \emoji{droplet} \\
        \hline
        G\textunderscore STIN & I\textunderscore OR & SIRO\textunderscore & AS\textunderscore O & V\textunderscore TOR \\
        \hline
    \end{tabular}
\end{center}

\begin{enumerate}
    \item 이야기
    \item 귤
    \item 달
    \item 가까운
    \item 집
\end{enumerate}

\pagebreak

\paragraph{3.}
빈칸에 들어갈 말로 가장 적절한 것은?

\begin{reminder}
    \begin{tabular}{ll}
        A & Elifashasas dofashas u iksùm a? \\
        B & Disáskéropo \textunderscore\ iksùm. \\
        A & Üpa ö vistùm!
    \end{tabular}
\end{reminder}

\begin{enumerate}
    \item u
    \item svo
    \item spo
    \item da
    \item lo
\end{enumerate}

\paragraph{4.}
빈칸에 순서대로 들어갈 말로 알맞은 것은?

\begin{reminder}
    \begin{itemize}
        \item Miros Naput lo \textunderscore.
        \item Sumeras \textunderscore.
    \end{itemize}
\end{reminder}

\begin{enumerate}
    \item koanùm, liviataicashaque
    \item koanùm, mavizash
    \item vulgatùm, liviataicashaque
    \item vulgatùm, mavizash
    \item karopùm, mavizash
\end{enumerate}

\paragraph{5.}
빈칸에 공통으로 들어갈 말로 알맞은 것은?

\begin{reminder}
    \begin{itemize}
        \item Lofafashas da \textunderscore.
        \item Miros \textunderscore\ lo Zasospika.
        \item Uspensénipotal \textunderscore\ Équorsinas öshùie uspensàz.
    \end{itemize}
\end{reminder}

\begin{enumerate}
    \item Potysin
    \item orüdùque
    \item orüdàs
    \item planariaùm
    \item malofanè
\end{enumerate}

\pagebreak

\paragraph{6.}
빈칸에 들어갈 말로 알맞은 것은?

\begin{reminder}
    \begin{tabular}{ll}
        A & Disicas Novéziroque lo taupùm. \\
          & Komòme, J Vais. \\
        B & Aniskas, J'Hakielysia svo komè Éie Vais izùm. \\
        C, D & \textunderscore.
    \end{tabular}
\end{reminder}

\begin{enumerate}
    \item Aniskas
    \item Aniska
    \item Üpa
    \item Vanishas
    \item Vanisha
\end{enumerate}

\paragraph{7.}
밑줄 친 부분과 바꾸어 쓸 수 있는 것은?

\begin{reminder}
    \begin{tabular}{ll}
        A & Lazverdie, \underline{Inofashas u doùie moqûm a?} \\
        B & A, miro degro narisàhum. \\
          & Lazverdie. \\
        A & Nah. Üpa ö vistùm!
    \end{tabular}
\end{reminder}

\begin{enumerate}
    \item Inofashas da ät a?
    \item Inofashas u komòme.
    \item Fashas d'Inos da iksùm a?
    \item Inofashasas dofashas a?
    \item Talisas fontlan a?
\end{enumerate}

\paragraph{8.}
빈칸에 들어갈 말로 알맞지 않은 것은?

\begin{reminder}
    \begin{tabular}{ll}
        A & Indromes niash u Miraésizin. \\
        B & Miro degro sihùie tagùm. \\
        A & Tia, \textunderscore.
    \end{tabular}
\end{reminder}

\begin{enumerate}
    \item Indromes fontlan
    \item niro degro indromàhome
    \item mòshainùm l'Indromes
    \item miros indromeùm
    \item nòindromê a
\end{enumerate}

\pagebreak

\paragraph{9.}
광고문을 이해한 내용으로 알맞은 것은?

\begin{reminder}
    \begin{center}
        \textbf{FONTLANÉVESIN LO MIGUTÙIE VAMOQUÒME!} \\
        Inofashas die Romin svo fontlan izùll sulah Évesinis lo migutùie nàvamoquòme.
        Vesin 16 les u T50, Hüliené 50\% roy to! \\
        \begin{tabular}{cc}
            Hanögäit & 2022 ärgê 23-tiéshetal \\
            Geshtalÿrgo & 0-5329-1234
        \end{tabular} \\
        9-Zôdésiros usuan geshtaluatùhum.
    \end{center}
\end{reminder}

\begin{enumerate}
    \item 9시 이전에만 전화할 수 있다.
    \item 행사는 2022년 6월 23일에 끝난다.
    \item 16개 50테로를 50\% 할인하여 판매한다.
    \item 로민 상점에서 진행하는 행사이다.
    \item 귤 할인 행사를 홍보하기 위한 광고이다.
\end{enumerate}

\paragraph{10.}
빈칸에 들어갈 말로 알맞은 것은?

\begin{reminder}
    \begin{tabular}{ll}
        A & Mirégeshtaluatin \textunderscore. \\
        B & Ät? Dash Novéskie a. \\
        A & Insürta, tia mòshainàhum.
    \end{tabular}
\end{reminder}

\begin{enumerate}
    \item da la' shain
    \item insürtaùie vudar
    \item lo Soriséskie roy uagizùdum
    \item da gamanùm
    \item lo vistòme
\end{enumerate}

\pagebreak

\paragraph{11.}
빈칸에 들어갈 말로 알맞은 것을 <보기>에서 찾아 순서대로 바르게 배열한 것은?

\begin{reminder}
    \begin{tabular}{ll}
        A & \textunderscore? \\
        B & Mirálonin. \textunderscore? \\
        A & Insürta!? Hiro samie Anroidélomialofafashas temkomè! \\
          & \textunderscore? \\
        B & Miros Tailofafashas u. \\
        A & Vequa a.
    \end{tabular}
\end{reminder}

\begin{reminder}
    \textbf{<보기>}
    \begin{enumerate}
        \item Dashas siro
        \item Nònarisùm lo hiro
        \item Niros doùie quovizashè a
    \end{enumerate}
\end{reminder}

\begin{enumerate}
    \item 1 - 2 - 3
    \item 1 - 3 - 2
    \item 2 - 1 - 3
    \item 2 - 3 - 1
    \item 3 - 1 - 2
\end{enumerate}

\paragraph{12.}
대화를 이해한 내용으로 알맞은 것은?

\begin{reminder}
    \begin{tabular}{ll}
        A & Disicégomédoçozaque u nòkomàd a? \\
        B & Mòvagùll lo Halumaka, Disicas Doliah u raduvùie moquàd. \\
          & Rifo tavla 9-Zôd ufas moquàz tou tagìç. \\
        A & Sih izùsua miro degro Hisézozaque u moquàd. \\
        B & Üpa. Fudis ät orüdùdum? \\
        A & Pizza nahùsua Salad da shain izàd e. \\
        B & Mòshainùm lo Pizza.
    \end{tabular}
\end{reminder}

\begin{enumerate}
    \item J'Axs Halumakal diosàhum.
    \item J'A ad J B's Disicémog u quovizashè.
    \item Raduvùsua 9–Zôdésiros u quovizashùhad.
    \item Gomorüdie royfas Pizza ad Salad l'orüdàd.
    \item J Bxs J'Axl shainùm.
\end{enumerate}

\pagebreak

\paragraph{13.}
빈칸에 들어갈 말로 알맞은 것은?

\begin{reminder}
    \begin{tabular}{ll}
        A & Sorisic u J Syul svo Ivaneçosiparadiç lo diosùm tou narisùm. \\
          & Nòmoquàd a? \\
        B & Nah, \textunderscore. \\
        A & Eçyàh, miro da moqùll kameraùll niro gie vizashàsad. \\
        B & Üpa.
    \end{tabular}
\end{reminder}

\begin{enumerate}
    \item mòmoqûm
    \item semüshain izàz
    \item dashas paradiç
    \item Disézeriquas shain
    \item Hisic u laz mashamaka
\end{enumerate}

\paragraph{14.}
빈칸에 들어갈 말로 알맞은 것만을 <보기>에서 고른 것은?

\begin{reminder}
    \begin{tabular}{ll}
        A & Niros Letuic u ät 'iosàd? \\
        B & \textunderscore. \\
    \end{tabular}
\end{reminder}

\begin{reminder}
    \textbf{<보기>}
    \begin{enumerate}
        \item Tasos flago tou tagùm.
        \item Foløzudia vizamoquàd.
        \item Talisas fontlan.
        \item Ät 'egro diosàhad.
    \end{enumerate}
\end{reminder}

\begin{enumerate}
    \item 1, 2
    \item 1, 3
    \item 2, 3
    \item 2, 4
    \item 3, 4
\end{enumerate}

\pagebreak

\paragraph{15.}
글을 이해한 내용으로 알맞지 않은 것은?

\begin{reminder}
    J'Hanguque svo Livialofafashas u Moquìl gorùllas, Sirosé 2-Lesémitodasé-nique é 1'l simùnum.

    1-Tia roy, Tailofafashas 'vo zurê Linfos l'apelùll, Pai' lo naritempùqûll Livialofafashas u Gamaçanis lo tempùque Émitoda.
    Disémitodas Tailofafashas 'vo zuràhusua poirnùhah.

    2-tia roy, LLVU Touatéskil uspensùll, Pai' lo naritempùqûll Livialofafashas u Gamaçanis lo tempùque Émitoda.
    Natio svo krifè Úspens izùll, Olélivialofafashas 'vo LLVU l'apelùll naristempùquê Épain roy Gamaçanis lo vistìç.
\end{reminder}

\begin{enumerate}
    \item 이 글은 한국에서 대학교에 가는 방법에 대한 내용을 다루고 있다.
    \item 첫 번째 방법은 고등학교에서 사용한 기록을 통해 점수를 계산하는 방법이다.
    \item 첫 번째 방법은 고등학교에 다니지 않으면 사용할 수 없다.
    \item 두 번째 방법을 사용하면 원서를 넣을 수 있는 대학이 한정된다.
    \item 수능 시험은 나라에서 운영하는 시험이다.
\end{enumerate}

\paragraph{16.}
글을 이해한 내용으로 알맞지 않은 것은?

\begin{reminder}
    Dise ufas, miros J Gusteauxévizashùqûaté ``Siro degro gustùhum" u kavùm Émirénoiz l'esleidàhe.
    Tia Disie ufas mònarisùm, lo His d'ät l'insürtaùie Dürziaèaim.
    Siro degro Ösulahédiazerizurin da veshùhahumia, Ösulahédiazerizurinas dofashas udegro tempùhum.
    Disie u J Gusteau svo gustìç Étiasut rad oh remie Équorsinas disatagùhah.
    Tia hiros, Disîginétag svofas, izùm J France ésiro rad ösulah Égustin.

    --- J'Anton Egox. Svo <La Ratatouille>
\end{reminder}

\begin{enumerate}
    \item J'Antonxas Dise udegro J Gusteauxl shainè.
    \item J'Antonxas J Gusteau u moqûll Fudil orüdè.
    \item J'Antonxas J Gusteau svo orüdè Éfudie da fontlan tou tagè.
    \item J'Antonxas J Gusteau svo Fudil orüdè Ésiros, Hasúaçitag\footnote{uaçitag: 가치관} lo uagizàse.
    \item J'Anton da vizashùm Ézozaque, J Gusteau svo gustìç Équorsinas remie.
\end{enumerate}

\pagebreak

\paragraph{17.}
`입국(Natiomodilìe)'에 관한 글을 읽고 이해한 내용으로 알맞지 않은 것은?

\begin{reminder}
    Dise ufas Zasoque u natiomodilùm Ézozaque u Ätésosalindy degro unasàhe. Tia Quinatio svo rivonùll Zasoque u komùll Has l'esleidùmaro, Quinatîn da Zasoque u modilùll Rivo' lo krifùmaro Évlazif da gaitevo ikomè. Rifo Erasheniluins dé Zasoqu'as, Durashtelo di'Amirvia Édurash die Media d'-Uatash Ésiros, Natiomodil'o gorùm Évizashosalindyl krifùll natiomodilùm Équorsinis u kavùm Élinfos lo potaùll Çynadìl tiägè.

    Erasheniluins dé Zasoque u Natiomodilìl gorùllas 1-Tia roy ä Vizashafashas u moqûll Hisénatámmiramitoda roy Sirin lo ammirasùnum. Siros, Vizashasin da Zasoquêrashenîkanuiz u uatempùll Modigamani' lo visenùm. Siros, Vizashasin da Modigamanin u uatempùll Modiquilas lo restorùm. Sih izùsua Modigamaninas Erasheniluins dé Zasoque u natiomodilùhum.
\end{reminder}

\begin{enumerate}
    \item 자소크 철학단에서는 곧 외국인 입국에 대한 입국 과정을 추가했다.
    \item 자소크 철학단에 입국하기 위해서는 일단 대사관에 찾아가야 한다.
    \item 대사가 직접 입국 희망자를 심사한 후 입국 희망자에게 연락한다.
    \item 과거에는 자소크 철학단에 입국하기 위해서 아무런 절차도 필요하지 않았다.
    \item 과거에 입국 절차가 없을 때에는 외국인들이 입국하여 사회적인 문제를 일으키기도 했다.
\end{enumerate}

\paragraph{18.}
글에서 설명하고 있는 도시는?

\begin{reminder}
    Disóires Erasheniluins dé Zasoque Áçentemoire.
    Disóires 2-Moréärge Ésoris svo Zasoque svo 1-roy Vizashùquóire izùll,
    Foløsulahésosaque u Erashenîka-nuiz da iksùm Óire izè.
\end{reminder}

\begin{enumerate}
    \item Zaägosh
    \item Hakielysia
    \item Piçevania
    \item Migesa
    \item Amaäs
\end{enumerate}

\pagebreak

\paragraph{19.}
문장 표현이 옳은 것만을 있는 대로 고른 것은?

\begin{reminder}
    \begin{enumerate}
        \item Miros Hanür l'orüde.
        \item J'Likixs Quorsin.
        \item Ad disas Zasospika.
    \end{enumerate}
\end{reminder}

\begin{enumerate}
    \item 1
    \item 3
    \item 1, 3
    \item 2, 3
    \item 1, 2, 3
\end{enumerate}

\paragraph{20.}
문장 표현이 틀린 것만을 있는 대로 고른 것은?

\begin{reminder}
    \begin{enumerate}
        \item Ovetiag anroidèfashas lo vizashìç Énirékafçie da shain.
        \item Uagizèsat svo nirésuanas sihùie iksòme.
        \item Matùie Nirol quovizamoqûlara.
    \end{enumerate}
\end{reminder}

\begin{enumerate}
    \item 2
    \item 3
    \item 2, 3
    \item 1, 3
    \item 1, 2, 3
\end{enumerate}

\begin{reminder}
    $\circ$ 수고하셨습니다. 문제지를 체출하시고 퇴실하셔도 좋습니다.
\end{reminder}

% 5 3 1 3 5 2 4 2 4 3 1 1 5 4 4 1 3 2 2 5

\end{document}

