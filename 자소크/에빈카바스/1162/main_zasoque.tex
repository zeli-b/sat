\documentclass{article}

\title{\textbf{ERASHENILUO}}
\author{1162-tié Evinkavas dé Vlazicamis}
\date{4313 ärge}

\begin{document}
    
\maketitle

4156 ärgêsoris to y Simetasis équorsinisas 70°U lo liviat éliviaciatashaqu-ímogas svo dotùne.
Y Simetasisxêrashenîkanuizas dis édot 'a y Simetasis é-quorsiné steligamalig l'aifidomùm tou tagùll, 11-tiéfatal, 4156-éfiu' lo tiägè.
Y Simetasisxêrashenîkanuizas Zasoque samie y Simetasis da priza anroidéshtelo-doliahin iz-touéskie, y Simetasis d'hülienùis Erasheniluins dé Zasoque éänonatio izè íe Evitagen spo tiägè touéskie, y Simetasis da ciamur étutayúr lo vagùsua hiséfashas u hülienùie dotè équorsinis da disocamàd touéskil olivizashùie tagùll fiunàd éskie da shain tou narisèz.

Erasheniluins dé Zasoque'as ä fiu' lo tiäg ézozaque, y Simetasis da tutayúsulahàs lo maravizashè touéskil narisùll Zasoquéuatashéquorsinis gie dis-ínsürtal narisàsull, natévenim lo poirn édisocaneistig l'ürpatè. Shitnataniùie y Simetasis samie Erasheniluins dé Zasoque éfiun svo olùie 80-gaie les éfiu-todisocamin da draviùll, 5000-gaie les édarutédisocamin da draviè.

\paragraph{Ironaris I.}
Has da tag ézozaque, y Simetasis équorsin az Erasheinluins dé Zasoque équorsinis gie vagùh ésemü svofas do d'iksis a?

\paragraph{Ironaris II.}
Has da tag ézozaque, Erasheniluins dé Zasoque équorsin az y Simetasisxéquorsinis gie vagùh ésemü svofas do d'iksis a?

\paragraph{Ironaris III.}
Erasheniluins dé Zasoque'as dézu' lo diosùne a?

\paragraph{Ironaris IV.}
Sulasatùie y Simetasis ézu' lo tagùsua, y Simetasis ézuras insürté-skie izè a?

\begin{center}
    — Lomiaè. Vlazicamis lo tempùqûll quiveshùhum. —
\end{center}

\end{document}